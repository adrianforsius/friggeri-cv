%!TEX TS-program = xelatex
\documentclass[]{friggeri-cv}
\addbibresource{bibliography.bib}

\begin{document}
\header{lorena}{matus}
       {International Business Enthusiast}

% In the aside, each new line forces a line break
\begin{aside}
  \section{Información}
    Brännkyrkagatan 119
    117 28 Stockholm
    Sweden
    ~
    \href{mailto:lorena-matus@hotmail.com}{lorena-matus@hotmail.com}
    +46 72 537 34 35
  \section{Idiomas}
    Español, Inglés
    Actualmente estudiando curso de sueco en Folkuniversitetet
  \section{Cualidades}
    Honesta \& Leal
    Responsable
    Aprendizaje rápido
    Trabajo bajo presión
    Agradable \& social 
    Trabajo en equipo
\end{aside}

\section{Resumen}
    Mi meta a corto plazo es terminar y perfeccionar el idioma sueco.
    Mi meta personal y profesional a largo plazo es crear mi propia empresa de import/export de productos mexicanos en Suecia para dar a conocer la riqueza de éste país.

\section{Educación}

\begin{entrylist}
  \entry
    {2006 - 2010}
    {Licenciado en Negocios Internacionales}
    {Con honores (B.Ib. cum laude)}
    {Universidad Autónoma de Baja California, Tijuana México.
    {\emph{Estudiante y Miembro del Comité Universitario.}}}
\end{entrylist}

\section{Experiencia}

\begin{entrylist}
  \bigentry
    {2014 - 2015}
    {ProMexico, Estocolmo, Suecia \newline 
        {\href{http://www.promexico.gob.mx/en/mx/stockholm}{promexico.gob.mx}}}
    {Practicante y Project Manager}
    {\emph{Durante el tiempo que estuve como practicante, colaboré en el departamento de importaciones y exportaciones. Mis tareas consistieron en realizar reportes de oportunidades de mercado para productos mexicanos en Suecia, reportes de oportunidades de inversión para empresas suecas en México, así como investigaciones y presentaciones para diferentes empresas.}}
    {\textit{ Durante el periodo en el que estuve realizando prácticas, se me consideró y contrató para sustituir a la persona encargada del Departamento de Exportaciones por un periodo de 3 meses de Junio - Agosto 2015.}}
  \bigentry
    {03–07 2014}
    {Restaurant Los Remedios, Tijuana, Mexico \newline
        {\href{http://www.losremedios.mx/}{losremedios.mx}}}
    {Hostess}
    {\emph{Encargada de servicio al cliente, reservaciones y eventos especiales.}}
    {\textit{ Durante la expansión de la franquicia y en base a mi desempeño, se me ofreció el puesto de Administración para el nuevo establecimiento.}}

  \entry
    {2011 - 2013}
    {High Chaparral, Hillerstorp, Sweden \newline
        {\href{http://www.highchaparral.se/sv/}{highchaparral.se}}}
    {Bailarina Profesional, Trabajo de verano}
    {\emph{Bailarina profesional de Folclore Mexicano para realizar el espectáculo mexicano en el parque temático High Chaparral al sur de Suecia. 
    Auxiliar a la coordinación del grupo de baile, ensayos y en eventos especiales realizados por el parque temático.}}
  \entry
    {03–05 2012}
    {World Trade Center, Tijuana, Mexico \newline
        {\href{http://www.wtctijuana.com/}{wtctijuana.com}}
    }
    {Organización}
    {\emph{Comité del evento ¨Encuentro Binacional entre México y Estados Unidos¨
    Auxiliar de logística, organización y realización de agenda de negocios para dicho evento.}}
  \bigentry
    {01–05 2011}
    {Vidriera Torre Aguacaliente S.A de C.V, Tijuana, Mexico}
    {Asistente administrativo}
    {\emph{Servicio al cliente, realización de cotizaciones, manejo de agenda de citas y entrega, pagos.}}
    {\textit{ Durante éste periodo diseñé un nuevo sistema de control de inventario el cual la empresa lo puso en práctica.}}
\end{entrylist}

\begin{entrylist}
  \entry
    {2008 - 2010}
    {Marmol Y Granito S.A de C.V, Tijuana, Mexico}
    {Asistente administrativo, Trabajo de verano}
    {\emph{Servicio al cliente, realización de cotizaciones, manejo de agenda de citas y entregas, pagos.}}
\end{entrylist}

\section{Experiencia académica}
\begin{entrylist}
  \entry
    {2006 - 2010}
    {Miembro del Comité Universitario}
    {Universidad Autónoma de Baja California, Tijuana}
    {Participación en los foros para la toma de decisiones respecto a los planes de estudio y aprobación del presupuesto universitario.}
  \entry
    {2009 - 2010}
    {Miembro del Comité de la Licenciatura en Negocios Internacionales}
    {Universidad Autónoma de Baja California, Tijuana}
    {Coordinadora general del evento ¨Semana de Negocios: Doing Business¨, para éste proyecto se contactaron a empresas locales del sector maquilador, consultoras y de gobierno.
    Éste proyecto se realizó en conjunto con ProMéxico, la oficina ubicada en tijuana B.C.
    El objetivo de éste proyecto fué visualizar el campo de acción para alumnos de Negocios Internacionales.}
\end{entrylist}

\section{Otras actividades}

\begin{entrylist}
  \entry
    {2014 - presente}
    {Bailiarina y maestra de Folclore Mexicano en Suecia}
    {Miembro del grupo México Lindo}
    {El grupo ensaya en Hornstull, donde soy maestra y bailarina de Folclore Mexicano.}
  \entry
    {2000 - 2014}
    {Bailarina y maestra de Folclore Mexicano}
    {́Miembro del Ballet Folclórico Yi Má}
    {Éste grupo pertenece a la Preparatoria Federal Lázaro Cárdenas en Tijuana, México.
    Miembro del equipo de baile que laboró en el parque temático High Chaparral en Suecia.}
  \entry
    {07-08 2007}
    {Bailarina}
    {́Miembro del Ballet Folclórico Yi Má}
    {Presentación en el festival de baile "Cork Folk
    Dance Festival" en Irlanda y Amsterdam.}
  \entry
    {07-08 2006}
    {Bailarina}
    {́Miembro del Ballet Folclórico Yi Má}
    {Presentación en el festival de baile "Kupalnocka" en Polonia.}
\end{entrylist}

\section{Intereses}
    Bailar, cocinar, maquillaje artístico y de fantasía, corte de cabello, eventos sociales, correr, viajar

\end{document}